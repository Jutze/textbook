\documentclass[a4paper,11pt]{book}
\usepackage[latin1]{inputenc}
\usepackage[english,ngerman]{babel}
%\usepackage{afterpage}
%\usepackage{booktabs}
%\usepackage{longtable}
\usepackage{graphicx}%\usepackage[draft]{graphicx}
%\usepackage{textcomp}%\usepackage{epic,eepic,ecltree}
%\usepackage{fancyhdr}
%\pagestyle{fancy}
%\lhead{}\chead{\leftmark}\rhead{}
%\lfoot{}\cfoot{}\rfoot{\thepage}
%\addtolength{\headheight}{3pt}
%\setlength{\LTcapwidth}{\linewidth}
%\addtolength{\headwidth}{\marginparsep}%\addtolength{\headwidth}{0.5\marginparwidth}
\usepackage{url}
\usepackage{hyperref}
\hypersetup{colorlinks=true,
linkcolor=black,
urlcolor=black,
citecolor=black,
pdfauthor={Johannes Schult},
pdftitle={Textbook},
pdfsubject={A not so gentle introduction to R, SPSS and Stata},
pdfkeywords={statistics, spss, stata, r, r-project, software, data analysis, datenanalyse, statistik, german, deutsch}}
\usepackage{apacite}
\bibliographystyle{apacite}
%%%in abstract = section no. 0 -> \setcounter{section}{-1}
%\linespread{1.24}
\title{R, Stata und SPSS, oder wie man Datenanalyse-Software verwendet, ohne den Verstand zu verlieren\thanks{Dies ist ein Arbeitstitel. Das Manuskript befindet sich noch im Aufbau.}}
\author{Johannes Schult\thanks{\url{jutze@jutze.com}}}
\date{Letzte �nderung am \today}
\begin{document}
\begin{titlepage}
	\maketitle
\end{titlepage}
%\setcounter{page}{0}
%\input{cover}
%\pagebreak
\pagebreak
\noindent Dieses Manuskript steht unter einer Creative-Commons-Lizenz Namensnennung--Nicht-kommerziell--Weitergabe unter gleichen Bedingungen (CC BY-NC-SA). Der genaue Lizenztext ist abrufbar unter \url{http://creativecommons.org/licenses/by-nc-sa/3.0}
\bigskip

\noindent Die Wiedergabe von Gebrauchsnamen, Handelsnamen, Warenbezeichnungen usw. in diesem Text berechtigt auch ohne besondere Kennzeichung nicht zu der Annahme, dass solche Namen im Sinne der Warenzeichen- und Marken\-schutz-Gesetzgebung als frei zu betrachten w�ren und daher von jederman benutzt werden d�rften.
\pagebreak
\tableofcontents
\input{vorwort}
%\pagebreak
%\listoffigures
%\listoftables
%\input{asdf.tex}
\input{einleitung}
\input{arbeitsfluss}
\input{computer}
\input{programme}
\input{daserstemal}
%\input{discussion.tex}

%\appendix
%\input{appendix.tex}

%\clearpage

\newpage
%\pagebreak

\bibliography{textbook}
\end{document}